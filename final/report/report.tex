%% This is file `elsarticle-template-1-num.tex',
%%
%% Copyright 2009 Elsevier Ltd
%%
%% This file is part of the 'Elsarticle Bundle'.
%% ---------------------------------------------
%%
%% It may be distributed under the conditions of the LaTeX Project Public
%% License, either version 1.2 of this license or (at your option) any
%% later version.  The latest version of this license is in
%%    http://www.latex-project.org/lppl.txt
%% and version 1.2 or later is part of all distributions of LaTeX
%% version 1999/12/01 or later.
%%
%% Template article for Elsevier's document class `elsarticle'
%% with numbered style bibliographic references
%%
%% $Id: elsarticle-template-1-num.tex 149 2009-10-08 05:01:15Z rishi $
%% $URL: http://lenova.river-valley.com/svn/elsbst/trunk/elsarticle-template-1-num.tex $
%%
\documentclass[preprint,12pt]{elsarticle}

%% Use the option review to obtain double line spacing
%% \documentclass[preprint,review,12pt]{elsarticle}

%% Use the options 1p,twocolumn; 3p; 3p,twocolumn; 5p; or 5p,twocolumn
%% for a journal layout:
%% \documentclass[final,1p,times]{elsarticle}
%% \documentclass[final,1p,times,twocolumn]{elsarticle}
%% \documentclass[final,3p,times]{elsarticle}
%% \documentclass[final,3p,times,twocolumn]{elsarticle}
%% \documentclass[final,5p,times]{elsarticle}
%% \documentclass[final,5p,times,twocolumn]{elsarticle}

%% The graphicx package provides the includegraphics command.
\usepackage{graphicx}
%% The amssymb package provides various useful mathematical symbols
\usepackage{amssymb}
%% The amsthm package provides extended theorem environments
%% \usepackage{amsthm}

\usepackage{lineno}
\usepackage{hyperref}

\journal{Universidad Tecnologica de pereira}

\begin{document}

\begin{frontmatter}

%% Title, authors and addresses

\title{Predict email opens}

\author{Manuel Felipe Pineda}

\address{Universidad Tecnologica de Pereira}

\begin{abstract}
  I this lab I will try to train a system with given
  metadata for emails sent to users in order to predict whether
  or not a future email will be opened for each user.

  This metadata contains specific information about:

  \begin{itemize}
    \item The user the email was sent to.
    \item The email that was sent.
    \item The user's reaction to the email, including
      (among other things) whether or not the user
      opened the email.
  \end{itemize}

  This metadata was taken from the emails sent to
  hackerrank users over a certain period of time.

  The proyect is based in a real competition that can be found
  \href{https://www.hackerrank.com/contests/machine-learning-codesprint/challenges/hackerrank-predict-email-opens}{here}
\end{abstract}

\begin{keyword}
Data Science \sep Machine learning \sep Complicated
%% keywords here, in the form: keyword \sep keyword

%% MSC codes here, in the form: \MSC code \sep code
%% or \MSC[2008] code \sep code (2000 is the default)

\end{keyword}

\end{frontmatter}

%%
%% Start line numbering here if you want
%%
% \linenumbers

%% main text
\section{split data}
\label{split:1}
random

\section{model validation}
- diferences

- F1 score

\section{strategies}
\label{strategy:1}

\subsection{Random}

- score with prob 0.5:
48625 differences: 0.497 \% per ok \\
F1 score 0.392 \\

- prob 0.7:
54999 differences: 0.431 \% per ok \\
F1 score 0.447

\subsection{counting}

53569 differences: 0.445 \% per ok \\
F1 score 0.443

\bibliographystyle{model1-num-names}
\bibliography{sample.bib}

%% Authors are advised to submit their bibtex database files. They are
%% requested to list a bibtex style file in the manuscript if they do
%% not want to use model1-num-names.bst.

%% References without bibTeX database:

% \begin{thebibliography}{00}

%% \bibitem must have the following form:
%%   \bibitem{key}...
%%

% \bibitem{}

% \end{thebibliography}


\end{document}

%%
%% End of file `elsarticle-template-1-num.tex'.
